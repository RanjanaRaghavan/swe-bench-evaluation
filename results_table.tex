\begin{table*}[t]
\centering
\caption{Generated Solutions for Software Engineering Problems}
\label{tab:results}
\begin{tabular}{|p{0.35\textwidth}|p{0.55\textwidth}|}
\hline
\textbf{Problem Description} & \textbf{Generated Solution} \\
\hline
{\footnotesize Fix the bug in PyTorch's \texttt{\_make\_tensor} function where it's not properly handling empty tensors with no elements.} & 
{\footnotesize
\begin{verbatim}
def _make_tensor(x, device): 
    if isinstance(x, torch.Tensor): 
        if device is not None: 
            x = x.to(device) 
        return x 
    if isinstance(x, (list, tuple)): 
        return torch.tensor([]).to(device) \
            if not x else torch.tensor(x, \
            device=device) 
    if isinstance(x, np.ndarray): 
        return torch.from_numpy(x).to(device) 
    return x
\end{verbatim}
} \\
\hline
{\footnotesize Fix the bug in the Transformers library where the \texttt{get\_special\_tokens\_mask} method doesn't properly handle already padded sequences.} & 
{\footnotesize
\begin{verbatim}
def get_special_tokens_mask(self, token_ids_0,
    token_ids_1=None, 
    already_has_special_tokens=False):
    if already_has_special_tokens:
        if token_ids_1 is not None:
            raise ValueError(
                "You should not supply a second " +
                "sequence if the provided " +
                "sequence of ids is already " +
                "formatted with special tokens.")
        return list(map(lambda x: 
            1 if self._is_padding_token(x) 
            else 0, token_ids_0))

    if token_ids_1 is None:
        return [1] + ([0] * len(token_ids_0)) + [1]
    return [1] + ([0] * (len(token_ids_0) + \
        len(token_ids_1))) + [1]

def _is_padding_token(self, token_id):
    return token_id == self.pad_token_id
\end{verbatim}
} \\
\hline
{\footnotesize Fix the bug in pandas where the \texttt{to\_numpy} method doesn't properly handle datetime64 arrays with mixed timezones.} & 
{\footnotesize
\begin{verbatim}
def to_numpy(self):
    if self.dtype.kind == 'M':
        tz-aware-ness = self.tz_localize()
        return np.asarray(tz-aware-ness.asi8)
    return np.asarray(self)
\end{verbatim}
} \\
\hline
{\footnotesize Fix the security vulnerability in the requests library where the SSL certificate verification can be bypassed.} & 
{\footnotesize
\begin{verbatim}
def merge_environment_settings(url, proxies, 
    stream, verify, cert):
    env_proxies = get_environ_proxies(
        url, no_proxy=None)
    merged_proxies = merge_setting(
        proxies, env_proxies)
    if not verify:
        verify = False
    merged_verify = merge_setting(verify, 
        os.getenv('REQUESTS_CA_BUNDLE', None))
    merged_cert = merge_setting(cert, None)

    return merged_proxies, merged_verify, \
        merged_cert
\end{verbatim}
} \\
\hline
{\footnotesize Fix the bug in FastAPI's dependency injection where optional dependencies with default values are not properly handled.} & 
{\footnotesize
\begin{verbatim}
async def solve_dependencies(
    *, request: Request, 
    dependant: Dependant,
    body: Optional[Union[Dict, Any]] = None
) -> Tuple[Dict, List[ErrorWrapper]]:
    values: Dict[str, Any] = {}
    errors: List[ErrorWrapper] = []
    for dependency in dependant.dependencies:
        if body is not None and \
            dependency.default is not None:
            values[dependency.name] = \
                body.get(dependency.name, 
                    dependency.default)
        elif dependency.cache:
            dependency_value = await \
                request.app.dependency_cache\
                .get_dependency_value(
                    dependency=dependency,
                    request=request,
                    body=body,
                )
        else:
            dependency_value = await \
                request.app.dependency_overrides\
                .get(dependency.call, 
                    dependency.default)\
                (request=request)
        values[dependency.name] = \
            dependency_value
    return values, errors
\end{verbatim}
} \\
\hline
\end{tabular}
\end{table*}
